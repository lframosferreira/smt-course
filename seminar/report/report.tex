\documentclass{article}

\usepackage{amsmath}
\usepackage{float}
\usepackage{graphicx}

\usepackage[colorlinks=true, allcolors=blue]{hyperref}

\title{Theory and Practice of SMT Solving - Seminar report}
\author{Luís Felipe Ramos Ferreira}

\begin{document}

\maketitle

The paper presents a novel tool called Z3-Noodler, which is a fork of the Z3 SMT theorem prover where the main string solver is substituted by
Noodler, a string theory solver based on the \textit{stabilization-based procedure} for solving string formulae. Since the procedure depends on the usage
of nondeterministic finite automata (NFA), the authors used a C++ library caled Mata for handling operations over automata.

Since the solver was developed over Z3, several tools inside it's environment were used, such as the SMT-LIB parser, formula preprocessing, linear integer arithmetic
solver, SAT solver, and also the whole DPLL(\(\mathcal{T}\)) architecture. In regards of the formula rewriting step, some modifications were
made in order to benefit the decision procedure used in Noodler. Some rewritings made by default were not benefitial for the new solver, so they were disbaled.
For example, rules that state the membership of a string term to a regular language are efficiently handled by Noodler, but they were initially preprocessed and changed
by the default Z3 solver.

In general, Noodler interacts with Z3 as follows. First, it receives a satisfying boolean assignment from Z3's SAT solver and then removes
useless assignments for dealing with a simpler formula. Noodler then converts the conjunction of string literals to a LIA constraint and feeds it back
to the solver as a theory lemma. Noodler also implements, as aforementioned, a decision procedure based on NFAs called \textit{stabilization-based procedure}.

Regarding the concepts around string theory handled by the solver, Noodler uses different ideas for each step. Noodler applies \textit{Axiom Saturation}, 
which saturates the input formula before the SAT solver starts generating assignments. This makes the solver avoid checking SAT assignments that are clearly false. Another
saturation, now regarding string predicates, is to replace some formulas with equivalent ones that are more suitable for the approach, such as length/regular
constraints. One example is the replacement of \(\neg\text{contains}(s, "abc")\) by \(s \notin \Sigma^*abc\Sigma^*\). 

The main decision procedure used by Noodler is the \textit{stabilization-based procedure}

During preprocesing, Noodler decreases the number of equations by applying conversion to regular constraints, propagation of epsilons (empty word), underapproximation rules, etc.
Another thing Noodler does is to check for trivial unsatisfiable parts of the formula for an eanly termination. Noolder current supports a lot of
string functions and predicates such as \texttt{replace}, \texttt{contains}, \texttt{substr}, \texttt{at}, \texttt{indexOf}, etc. The predicates \texttt{replace\_all} and
\texttt{to/from\_int} are not supported. In general, the decision procedure implemented in Z3-Noodler can handle the fragment of string theory of chain-free fragment with unbounded
disequations and regular constraints, and quadratic equations. 

The results presented in the paper show that Z3-Noolder is capable of competing against other string solvers in the state of the art.



\end{document}
