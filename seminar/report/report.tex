\documentclass{article}

\usepackage{amsmath}
\usepackage{float}
\usepackage{graphicx}

\usepackage[colorlinks=true, allcolors=blue]{hyperref}

\title{Theory and Practice of SMT Solving - Seminar report}
\author{Luís Felipe Ramos Ferreira}

\begin{document}

\maketitle

The paper presents a novel tool called Z3-Noodler, which is a fork of the Z3 SMT theorem prover where the main string solver is substituted by
Noodler, a string theory solver based on the \textit{stabilization-based procedure} for solving string formulae. Since the procedure depends on the usage
of nondeterministic finite automata (NFA), the authors used a C++ library caled Mata for handling operations over automata.

Since the solver was developed over Z3, several tools inside it's environment were used, such as the SMT-LIB parser, formula preprocessing, linear integer arithmetic
solver, SAT solver, and also the whole DPLL(\(\mathcal{T}\)) architecture. In regards of the formula rewriting step, some modifications were
made in order to benefit the decision procedure used in Noodler. Some rewritings made by default were not benefitial for the new solver, so they were disbaled.
For example, rules that state the membership of a string term to a regular language are efficiently handled by Noodler, but they were initially preprocessed and changed
by the default Z3 solver.

In general, Noodler interacts with Z3 as follows. First, it receives a satisfying boolean assignment from Z3's SAT solver and then removes
useless assignments for dealing with a simpler formula. Nooler then converts the conjunction of string 




\end{document}
